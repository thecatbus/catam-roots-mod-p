\documentclass[12pt]{article}

\usepackage{titlesec}
\usepackage{enumerate}
\usepackage[parfill]{parskip}
\usepackage{amsmath}
\usepackage{amssymb}
\usepackage{amsthm}
\usepackage{caption}
\usepackage{subcaption}
\usepackage{float}
\usepackage{graphicx}
\usepackage{verbatim}

\titleformat{\section}
    {\normalfont\fontsize{15}{17}\bfseries}{\thesection}{2.27em}{}
\titleformat{\subsection}
    {\normalfont\fontsize{17}{18}\bfseries}{\thesubsection}{1em}{}
\titleformat{\subsubsection}
    {\normalfont\fontsize{15}{15}\bfseries}{\arabic{subsubsection}}{1em}{}

\newcommand{\starreditem}{\item[\refstepcounter{enumi}\number\value{enumi}*.]}
\newcommand{\textmod}[1]{\;(\text{mod }#1)}
\newcommand{\mathmod}[1]{\quad(\text{mod }#1)}

\begin{document}
\setcounter{section}{14}
\section{Number Theory}
\setcounter{subsection}{5}
\subsection{Computing roots Modulo \textit{p}}
\emph{CATAM coursework for Part II of the Mathematical Tripos. Sections have
been numbered as they appear in the manual.}

\setcounter{subsubsection}{1}
\subsubsection{Computing Legendre Symbols}
\textbf{Question 1} \quad 
We implement the repeated squaring method for modular exponentiation using a
recursive algorithm: 
\[a^n \equiv \begin{cases}
    (a^{\frac{n}{2}})^2, &n\text{ even}, n>0\\
    a\cdot (a^{\frac{n-1}{2}})^2, &n\text{ odd}\\
    1, &n = 0
\end{cases} \quad (\text{mod } p)\]
This allows for efficient application of Euler's criterion to compute Legendre
symbols. For \(p=30275233\), we compute \((a/p)\) for \(a\) taking:
\begin{enumerate}[(i)]
    \item \(100\) random values between \(1\) and \(p\). Out of these, ?? numbers
        were found to be quadratic residues mod \(p\).
    \item all values between \(1\) and \(100\). Out of these, \(58\) numbers
        were found to be quadratic residues mod \(p\).
\end{enumerate}
Appendix A contains a record of all output produced.

 \hfill

\textbf{Question 2}\quad Suppose \(n\) is any odd number and \(m\in
\mathbb{Z}\).  If \(n=1\), we have \((m/n)=(m/1)=1\). Otherwise, we may assume
\(0\leq m < n\) by reducing mod \(n\) if necessary. Note that \((0/n)=0\) when
\(n>1\). If \(m\) is a non-zero even number, we may get rid of
all factors of \(2\) using the following property of the Jacobi symbol:

\textbf{Lemma} For any positive odd \(n\) we have
\[\left(\frac{2}{n}\right) = (-1)^{(n^2-1)/8}= \begin{cases} +1, &n \equiv \pm1
    \mathmod{8} \\ 
-1, &n\equiv \pm 3 \mathmod{8}\end{cases}\]

A proof of this can be found in [1] (\textit{pp.} 47, Proposition
II.2.6). Hence we may assume \(m,n\) are both odd and \(1\leq m<n\). We state the following strengthening of quadratic reciprocity: 

\textbf{Lemma} For any two positive odd integers \(m\) and \(n\) we have
\[\left(\frac{m}{n}\right) = (-1)^{(m-1)(n-1)/4}\left(\frac{n}{m}\right).\]

A proof of this can be found in [1] (\textit{pp.} 47, Proposition
II.2.7), but it follows immediately from the law of quadratic reciprocity and
the observation that \((m/n)=0\) whenever \(m\) and \(n\) are not coprime.
Moreover, we can avoid computing the large product \((m-1)(n-1)/4\) by observing
that 
\[(-1)^{(m-1)(n-1)/4} = \begin{cases}
    -1, & m,n\equiv 3 \quad(\text{mod }4) \\
    +1, &\text{otherwise}
\end{cases}\] 

To compute \((m/n)\), it now suffices to compute \((n/m)\). Note that the
`denominator' strictly decreases after each iteration, hence the recursion must
halt in finite time.

COMPLEXITYCOMPLEXITYCOMPLEXITY

\subsubsection{Computing square roots mod \textit{p}}
\textbf{Question 3}\quad 
Suppose \(p\equiv 3\;(\text{mod }4)\) and \(a\) is a residue mod \(p\). In
particular, Euler's criterion implies \(a^{(p-1)/2}\equiv 1\textmod{p}\). Then \(x\equiv
a^{(p+1)/4}\) is a solution to \(x^2\equiv a\textmod{p}\) since
\[x^2 \equiv a^{(p+1)/2} \equiv a\cdot a^{(p-1)/2} \equiv a \mathmod{p}.\]

Suppose \(p\equiv 5\textmod{8}\). We have \((2/p) \equiv
2^{(p-1)/2}\equiv -1 \textmod{p}\). If \(a\) is a quadratic residue, we have
\(a^{(p-1)/2}\equiv 1 \Rightarrow a^{(p-1)/4}\equiv \pm 1 \textmod{p}\). In
particular, we can write \( 2^{k(p-1)/2} \cdot a^{(p-1)/4}\equiv 1 \textmod{p}\) for some \(k\in
\{0,1\}\). But then we have \(a \equiv 2^{k(p-1)/2} \cdot a^{(p-1)/4}\cdot a \equiv
2^{k(p-1)/2}\cdot a^{(p+3)/4} \textmod{p}\). Observe that all the exponents
involved are even, so we can read off a solution to \(x^2\equiv a\) as \(x\equiv
2^{k(p-1)/4}\cdot a^{(p+3)/8}\textmod{p}\).

\hfill 

Now suppose \(p\) is a prime of the form \(2^n+1\) (we only consider \(n>2\)
since \(p=3,5\) have been covered above). Then \(p\equiv (-1)^n+1\textmod{3}\),
hence \(n\) must be even. It follows that \(p\equiv 1\textmod{4}\) and \(p\equiv
2\textmod{3}\), so that \((3/p) = (p/3) = (2/3) = -1\). Let \(g\) be any
primitive root mod \(p\). The subgroup \(\langle g^2\rangle\) has order
\(2^{n-1}=\frac{p-1}{2}\) hence contains all the quadratic residues. Moreover,
it is the unique multiplicative subgroup of order \(2^{n-1}\). Since \(3\notin
\langle g^2\rangle\), the multiplicative order of \(3\) must be \(2^n\) i.e.\
\(3\) is also a primitive root mod \(p\). 

\textbf{Question 4}\quad
Suppose \(p=65537=2^{16}+1\), and we wish to solve the congruence \(x^2\equiv
18612 \textmod{p}\). We compute \((18612/65537)=1\) using the program written
for Question 2, so such an \(x\) exists. For the purposes of this question, use
`\(=\)' to denote congruence mod \(p\). For modular exponentiation, we use a
program based on the repeated squaring method.

Since \(3\) is a primitive root mod \(p\), we may write \(x =
3^{r_0+2r_1+2^2r_2+...}\) where each \(r_j\in \{0,1\}\). The congruence
\(x^2=18612\) can be written as \(\prod_{j\geq 0} 3^{r_j2^{j+1}} = 18612
\textmod{p}\). 

Raising both the sides to \(2^{14}\), all the terms for \(j\geq 1\) vanish and
we are left with \(3^{r_02^{15}}= 18612^{2^{14}} = 1\). Since \(3\) is a
primitive root, \(3^{2^{15}} = -1\) by Euler's criterion and we have \(r_0=0\).
In fact, \(18612^{2^n}=1\) for all \(14\geq n\geq 11\) hence we have \(r_0=...=r_3 =
0\). 

Raising both the sides of \(\prod_{j\geq 4}3^{r_j2^{j+1}}=18612\) to the power
\(2^{10}\), we obtain \(3^{r_42^{15}}=-1\), hence \(r_4 = 1\). We may multiply
both the sides of the congruence by \(3^{-r_42^5} = 3^{2^{16}-2^5} = 29606\) to
obtain \(\prod_{j\geq 5}3^{r_j2^{j+1}}=57313.\) 

Again, \(57313^{2^n}=1\) for \(9\geq n \geq 6\), so \(r_5=...=r_8=0\), and we
have \(\sum_{j\geq 9}3^{r_j2^{j+1}} = 57313\). Raising both the sides to the
power \(2^5\), we obtain \(3^{r_92^{15}}=-1\), hence \(r_9 = 1\). Multiply both
the sides by \(3^{-r_92^{10}} = 3^{2^{16}-2^{10}} = 64509\) to obtain
\(\prod_{j\geq 10}3^{r_j2^j} = 65536 = -1\). Comparing with \(3^{2^{15}} = -1\),
we deduce \(r_{10} = ... = r_{13}=0\) and \(r_{14}=1\).

We can now read off the solution to \(x^2 = 18612\) as \(x= 3^r\) where
\(r=2^4+2^9+2^{14}\). A square root of \(18612\) (mod \(p\)) hence is
\(45462\). The other square root is \(-45462= 20075\) and these are the only
solutions to \(x^2 =18612\) since \((\mathbb{Z}/p\mathbb{Z})^\ast\) is a field.

\hfill 

Suppose \(p\) is any odd prime and \(a\) is a quadratic residue mod \(p\). In
this section, use `\(=\)' to denote congruence mod \(p\). We may find \(\alpha >
0\) and \(s\) odd such that \(p-1 = 2^\alpha s\). Since \(s\) is odd, we
define \(z = a^{(s+1)/2}\) and observe that if \(y^2 = a^s\), then \(zy^{-1}\)
is a square root of \(a\) mod \(p\) (where \(y^{-1}\) is the multiplicative inverse of
\(y\) in \((\mathbb{Z}/p\mathbb{Z})^\times\).)

Now \((a^s)^{2^{\alpha - 1}} = a^{(p-1)/2} = (a/p)= 1\), so \(y\) is an  element of the
cyclic multiplicative group \(G=\{g\in
(\mathbb{Z}/p\mathbb{Z})^\times\;|\;g^{2^{\alpha}} = 1\}\). Suppose \(n\) is any
non-residue mod \(p\), and let \(b=n^s\). Then we have \(b^{2^\alpha}=n^{p-1} =
1\) hence \(b\in G\), and moreover \(b\) generates the group since
\(b^{2^{\alpha - 1}}=n^{(p-1)/2} = (n/p)=-1\). We can then write \(y = b^r\),
and solve for \(r\) algorithmically by considering its binary expansion. A
square root of \(a\), then, is \(zb^{2^{\alpha}-r}\).

\textbf{Question 5}\quad We implement an algorithm that uses the above method to
compute square roots mod \(p\) when \(p\equiv 1 \textmod{8}\), using the more
direct computations from Question 3 to handle other cases. The second solution
to the congruence \(x^2\equiv a \textmod{p}\) can be computed as \(-1\) times the first
solution. 

Here are some test cases, in the form \texttt{(a, root a)}. The values of \(a\)
(other than the first) have been chosen randomly subject to being quadratic
residues. The primes have been chosen\footnote{The first prime is the largest
known Fermat prime, while the latter four are relatively large palindromic
primes [OEIS: A055578].} to cover all possible congruence classes mod \(8\). The
generated roots can be confirmed to be accurate by squaring them. In particular,
the first test case agrees with what we found in Question 4.

\verbatiminput{../output/Test-roots.txt}

Additionally, we compute the roots of all quadratic residues in
\(\{1,2,...,20\}\) mod \(30275233\) and present the results in Appendix B.

COMPLEXITYCOMPLEXITYCOMPLEXITY

\subsubsection{Computing roots of polynomials mod \textit{p}}

\subsubsection*{Appendix A: Computed Legendre symbols}
Using Euler's criterion, Legendre symbols were computed for \(100\) random
values between \(1\) and \(p=30275233\). The results are recorded here in the
format \(\texttt{a: (a/p)}\).

\verbatiminput{../output/Legendre-random.txt} 

We also perform a similar calculation for all \(a\) between \(1\) and \(100\).

\verbatiminput{../output/Legendre-100.txt} 

\subsubsection*{Appendix B: Roots of residues mod 30275233}
For every integer \(1\leq a\leq 20\), we check if \(a\) is a quadratic residue
mod \(p=30275233\) and if so, compute a solution to \(x^2\equiv a\textmod{p}\).
The computed roots are presented in format \texttt{(a, root a)}.

\verbatiminput{../output/Twenty-roots.txt} 

\subsubsection*{References}
\begin{enumerate}[{[}1{]}]
    \item Koblitz, N. \textit{A course in Number Theory and Cryptography},
        Graduate Texts in Mathematics 114, Springer, 1987.
\end{enumerate}
\end{document}
